%! Author = Len Washington III
%! Date = 9/15/25

% Preamble
\documentclass[
	lecture={8},
	title={Mineral Resources}
]{enve201notes}

% Packages

% Document
\begin{document}

\setcounter{chapter}{7}
%<*Lecture-8>
\chapter{Mineral Resources}\label{ch:mineral-resources}
\begin{itemize}
	\item Geologists divide mineral resources into two categories.
	\begin{description}
		\item[Metallic mineral resources] include gold, copper, aluminum, iron, or other metals.
		\item[Nonmetallic mineral resources] include building stone, gravel, sand, gypsum, phosphate, and salt, used in construction or for chemical production.
	\end{description}
	\item \emph{What is a metal?}
\end{itemize}

\section{Metal}\label{sec:metal}
\begin{itemize}
	\item Metals are opaque, shiny, and smooth solids that conduct electricity.
	\item These properties come from metallic bonds, in which electrons are delocalized - free to move easily from atom to atom.
	\begin{itemize}
		\item In contrast, covalent or ionic bonds in nonmetallic materials do not allow this kind of electron mobility.
	\end{itemize}
	\item The behavior of a metal depends on:
	\begin{itemize}
		\item The strength of bond between atoms;
		\item The architecture of its crystal structure.
	\end{itemize}
	\item Metals are also known for their mechanical properties:
	\begin{description}
		\item[Ductility] the ability to be drawn into thin wires.
		\item[Malleability] the ability to be hammered into thin sheets.
	\end{description}
\end{itemize}

\section{Discovery of Metals}\label{sec:discovery-of-metals}
\subsection{Early Metals}\label{subsec:early-metals}
\begin{description}
	\item[Gold (ca. 6,000 BCE)] likely the first humans encountered \bc\ it occurs naturally in pure (native) form.
	\item[Copper (ca. 5,000 BCE)] first metal smelted from ores.
	Initially used in pure form, but too soft for durable tools.
	\item[Bronze (ca. 3,300 BCE)] discovery of alloying copper \w\ tin created bronze, which is stronger and harder.
	Marked a major leap in tool and weapon production, leading to the Bronze Age.
\end{description}

\subsection{Iron Age}\label{subsec:iron-age}
\begin{description}
	\item[Iron (ca. 1,200 BCE)] more abundant than copper and tin, but harder to smelt (requires higher temperatures).
	Compared \w\ copper or bronze, iron offered greater strength, hardness, and abundance, eventually replacing bronze for tools, weapons, and building materials.
	\item[Steep (ca. BCE, widely used by $\sim$1,000 CE)] an alloy of iron and carbon, stronger and more versatile that pure iron.
	\item[Stainless steel (1,913 CE)] alloy of iron \w\ chromium, highly resistant to corrosion.
	Revolutionized modern engineering, especially in medicine, food, and infrastructure.
\end{description}

\subsection{Modern Metals}\label{subsec:modern-metals}
\begin{description}
	\item[Aluminum (1,825 CE)] lightweight, corrosion-resistant; now essential for aviation, packaging, and construction.
\end{description}

\subsubsection{Other 19th-20th Century Discoveries Include:}{subsubsec:20th-cent-metals}
\begin{description}
	\item[Titanium (1791)] lightweight, strong, corrosion-resistant.
	\item[Uranium (1789)] key for nuclear energy in the 20th century.
	\item[Rare Earths (1790s-1900s)] a set of 17 nearly indistinguishable lustrous silvery-white soft heavy metals.
	Now critical for electronics and green energy.
	\begin{itemize}
		\item Scandium (\ce{Sc})
		\item Yttrium (\ce{Y})
		\item Cerium (\ce{Ce})
		\item Neodymium (\ce{Nd})
		\item Lanthanum (\ce{La})
		\item Dysprosium (\ce{Dy})
		\item Samarium (\ce{Sm})
		\item Terbium (\ce{Tb})
		\item Praseodymium (\ce{Pr})
		\item Gadolinium (\ce{Gd})
		\item Europium (\ce{Eu})
		\item Ytterbium (\ce{Yb})
		\item Holmium (\ce{Ho})
		\item Lutetium (\ce{Lu})
		\item Thulium (\ce{Tm})
		\item Erbium (\ce{Er})
		\item Promethium (\ce{Pm})
	\end{itemize}
\end{description}

\section{Native Metals}\label{sec:native-metals}
\begin{itemize}
	\item Native metals (gold, silver, copper, iron) occur naturally in their pure, metallic form, rather than as part of a compound.
	\item Native metals are relatively rare in nature, which is why they are so valuable.
	In fact, native metals make up only a tiny fraction of the world's current metal supply.
\end{itemize}

\section{Ore Minerals}\label{sec:ore-minerals}
\begin{itemize}
	\item Naturally occurring minerals that contain a sufficient concentration of valuable elements, typically metals, to be extracted and processed.
	\item Ores are often associated \w\ gangue minerals, which are non-valuable materials that need to be separated from the desired ore minerals.
	\item Principal metals in most common use today:
	\begin{itemize}
		\item Aluminum from Bauxite
		\item Iron from Hematite and Magnetite
		\item Copper from Chalcopyrite
		\item Zinc from Sphalerite
		\item Lead from Galena
	\end{itemize}
\end{itemize}

\section{Smelting}\label{sec:smelting}
\begin{itemize}
	\item A high-temperature process that extracts a metal from its ore.
	\item When minerals decompose during smelting, the process yields a metal (useful product) and a nonmetallic residue known as slag.
	\item Different ores require different smelting techniques at varying temperatures.
\end{itemize}

\subsection{Smelting vs Melting}\label{subsec:smelting-vs-melting}
\begin{itemize}
	\item Melting involves heating a substance to its melting point, causing it to change from a solid to a liquid \wo\ a chemical change.
	\item Smelting is a more complex process that includes melting but also involves a chemical reaction to remove oxygen and separate the metal from impurities.
\end{itemize}

\section{Ore-forming Processes}\label{sec:ore-forming-processes}
\begin{itemize}
	\item Ore deposits typically require specialized geological processes to concentrate valuable metals or minerals into economically recoverable amounts.
	\begin{itemize}
		\item Magmatic processes
		\item Hydrothermal processes
		\item Metamorphic processes
		\item Sedimentary processes
	\end{itemize}
	\item Whereas, other deposits form by more common geologic processes like erosion, deposition, or evaporation.
	\item c.e., metals and many useful minerals are typically present in the Earth's crust in very low concentrations.
	To make mining economical, nature must concentrate them far beyond their average abundance!
\end{itemize}

\section{Ore Deposits}\label{sec:ore-deposits}
\begin{description}
	\item[Magmatic process] as magma cools, certain materials crystallize earlier or segregate into dense layers.
	This can concentrate metals like chromium, nickel, or platinum.
	\item[Hydrothermal process] hot fluids dissolve metals, travel through fractures, and precipitate valuable minerals in veins when conditions (temperature, pressure, chemistry) change.
	\item[Metamorphic process] metamorphism can remobilize and re-concentrate minerals under high pressure and temperature.
	\item[Sedimentary process] metals like iron, manganese, or uranium can concentrate in chemical sedimentary rocks (e.g., banded iron formations).
	\begin{itemize}
		\item Banded iron formations (BIFs) consist of alternating layers of gray hematite (\ce{Fe2O3}) and iron-rich red chert (jasper).
		\item Manganese nodules grow slowly on the sea floor and are rich in \ce{MnO2} and trace elements.
	\end{itemize}
\end{description}

\section{Ore Exploration and Production}\label{sec:ore-exploration-and-production}
\begin{description}
	\item[Prospecting] geologists and prospectors look for visible traces of ore minerals in rock outcrops.
	\begin{description}
		\item[Geological surveys] measurements of magnetism, gravity, or radioactivity can reveal anomalies in the Earth's field that may indicate hidden ore bodies.
		\item[Geochemical surveys] soil, water, sediments, and even plants (biota) are analyzed.
	\end{description}
	\item[Drilling and sampling] once an ore deposit is suspected, geologists drill boreholes to collect subsurface rock samples.
	This helps determine the shape, size, and extent of the deposit.
	\item[Production (mining)] the type of mining depends on the proximity of the ore body to the surface.
	\begin{description}
		\item[Open-pit mining] used when ore bodies are close to the surface.
		It involves removing overburden and excavating the ore in large benches.
		This method is cheaper and safer.
		\item[Underground mining] necessary for deeper ore bodies.
		It involves shafts and tunnels to access ore.
		Although it reaches greater depths, it is more expensive and hazardous, \w\ risks like tunnel collapses, poisonous gases, and explosions.
	\end{description}
\end{description}

\section{Mining and Ore Processing}\label{sec:mining-and-ore-processing}
\begin{itemize}
	\item Miners extract crude ore (yellow)
	\item Waste rock is often piled in waste rock dumps.
	\begin{itemize}
		\item Not all rock removed contains valuable minerals.
		\item Waste rock is the unmineralized or low-grade rock that must be separated and dumped.
	\end{itemize}
	\item The crude ore is transported to processing facilities
	\item Ore goes to concentration and chemical separation ponds
	\begin{itemize}
		\item Valuable minerals are separated from waste through flotation, leaching, or other chemical methods.
	\end{itemize}
	\item After separation, the valuable portion is collected as concentrated metal.
	\item The leftover slurry of ground rock and processing chemicals is called tailings.
	\begin{itemize}
		\item Tailings are stored in tailings ponds, often held back by dams to prevent them from flowing downstream.
		\item These storage areas are major environmental concerns, as failures can release toxic material.
	\end{itemize}
\end{itemize}

\section{Mining and Environment}\label{sec:mining-and-environment}
\begin{itemize}
	\item Mineral extraction and processing leaves significant ecological footprints.
\end{itemize}

\subsection{Land disturbance}\label{subsec:land-disturbance}
\begin{itemize}
	\item Mining reshapes landscapes through open pits, waste dumps, and tailing ponds.
	\item Removal of vegetation and topsoil leads to erosion and loss of biodiversity.
\end{itemize}


\subsection{Waste Rock and Tailings}\label{subsec:waste-rock-and-tailings}
\begin{itemize}
	\item Waste rock piles may leach harmful chemicals, especially if they contain sulfide minerals (leading to acid mine drainage).
	\item Tailings pongs store toxic slurries of fine particles mixed \w\ processing chemicals.
	Dam failures have caused catastrophic downstream destruction.
\end{itemize}

\subsection{Water Pollution}\label{subsec:water-pollution}
\begin{itemize}
	\item Acid Mine Drainage (AMD): When sulfide-rich waste reacts \w\ water and oxygen, it produces sulfuric acid, dissolving heavy metals (like arsenic, lead, and mercury) into rivers and groundwater.
	\item Cyanide and mercury contamination from gold mining can poison ecosystems.
\end{itemize}

\subsection{Air Pollution}\label{subsec:air-pollution}
\begin{itemize}
	\item Dust from blasting, hauling, and crushing contributes to particulate air pollution.
	\item Smelting releases sulfur dioxide (\ce{SO2}), which can form acid rain.
	\item GHG from fuel use and processing add to climate change impacts.
\end{itemize}

\section{Nonmetallic Rock and Mineral Resources}\label{sec:nonmetallic-rock-and-mineral-resources}
\begin{itemize}
	\item Common nonmetallic resources are used in construction, agriculture, industry, and chemical processes.
	\begin{itemize}
		\item e.g., limestone, crushed stone, granite, marble, gypsum, phosphate, pumice, clay, and, salt, and sulfur.
	\end{itemize}
	\item Dimension stone: intact slabs and blocks of rock
	\begin{itemize}
		\item e.g., granite, marble, limestone
		\item Workers carefully cut rock from quarry walls to preserve intact pieces.
	\end{itemize}
	\item Crushed stone (aggregate): broken rock used as raw material for cement, concrete, and asphalt.
	\item Concrete and mortar: rock-like mixtures made of aggregate (sand, gravel, or crushed rock) held together by cement.
	\item Cement is the powder that consists of lime (\ce{CaO}), quartz (\ce{SiO2}), aluminum oxide (\ce{Al2O3}), and iron oxide (\ce{Fe2O3}).
	It acts as the binder in concrete and mortar.
	\begin{itemize}
		\item Natural cement was first developed by the Ancient Romans and is made from volcanic ash + limestone heated in kilns.
		Rarely produced today.
		\item Portland cement is most common modern cement.
		It is made by carefully mixing crushed limestone, sandstone, and shale in correct proportions.
	\end{itemize}
\end{itemize}

\section{Nonmetallic Minerals for Homes and Farms}\label{sec:nonmetallic-minerals-for-homes-and-farms}
\begin{description}
	\item[Clay] forms from the chemical weathering of silicate rocks.
	It is used for bricks, pottery, porcelain, and other ceramics.
	\item[Quartz (silica)] sourced from quartz sand (beach or sandstone).
	It forms glass and is also used in photovoltaic cells for solar panels.
	\item[Gypsum] mixed \w\ water into a slurry and layered paper.
	It produces gypsum board (drywall) for building construction.
	\item[Plastics] derives from oil extracted from underground reserves.
	\item[Asbestos] come from serpentine, formed by the reaction of olivine \w\ water.
	Asbestos was historically used for fireproofing and insulation.
\end{description}

\section{How Long will Resources Last?}\label{sec:how-long-will-resources-last?}
\begin{itemize}
	\item Mineral resources are nonrenewable - once an ore deposit or a limestone hill is mined, it's gone forever.
	\item Industrialized countries consume vast amounts of mineral resources each year.
	\item Strategic minerals are of particular concern: manganese, platinum, chromium, and cobalt.
	These metals are alloyed \w\ iron to make special-purpose sheets (e.g., aerospace industry).
\end{itemize}

%</Lecture-8>
\end{document}
