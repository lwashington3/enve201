%! Author = Len Washington III
%! Date = 9/3/25

% Preamble
\documentclass[
	lecture={5},
	title={Plate Tectonics}
]{enve201notes}

% Document
\begin{document}

\setcounter{chapter}{4}
%<*Lecture-5>
\chapter{Plate Tectonics}\label{ch:plate-tectonics}
\section{Theory of Plate Tectonics}\label{sec:theory-of-plate-tectonics}
\begin{itemize}
	\item The outer layer of the Earth, called the \emph{lithosphere}, is broken into large rigid pieces known as tectonic plates.
	These plates move slowly over the softer, ductile \emph{asthenosphere} beneath them.
	\item In geology, the development of plate tectonic theory marked a true scientific revolution.
\end{itemize}

\section{Basic Principles of Plate Tectonics Theory}\label{sec:basic-principles-of-plate-tectonics-theory}
\begin{itemize}
	\item The Earth's lithosphere is divided into plates that move relative to one another.
	\item The motion of one plate relative to its neighbor takes place by slip along plate boundaries.
	\item Continents are parts of some plates, so as the plates move, the continents are carried along \w\ them.
	\item \Bc\ of plate tectonics, the Earth's surface is dynamic--the positions of continents and oceans change continuously over geologic time.
\end{itemize}

\section{Lithosphere and Asthenosphere}\label{sec:lithosphere-and-asthenosphere}
\subsection{Lithosphere}\label{subsec:lithosphere}
\begin{itemize}
	\item Includes of the crust and the uppermost part of the upper mantle.
	\item Acts as a relatively rigid, hard layer.
	When subjected to stress, it does not flow but instead bends (elastic deformation) or breaks (faulting, fracturing).
\end{itemize}

\subsection{Asthenosphere}\label{subsec:asthenosphere}
\begin{itemize}
	\item Made of upper mantle that is hotter and more ductile than the lithosphere.
	\item Can flow slowly when stressed, similar to convection in boiling water, but on geological timescales (million of years).
\end{itemize}

\section{Lithosphere Plates}\label{sec:lithosphere-plates}
\begin{itemize}
	\item The lithosphere is broken into $\sim20$ discrete plates.
	\begin{description}
		\item[Major plates (12)]: Cover large surface areas (e.g., Pacific, African plates).
		\item[Microplates]: Smaller plates, often along plate boundaries or fragmented regions.
	\end{description}
\end{itemize}

\subsection{Plate Boundaries}\label{subsec:plate-boundaries}
\begin{itemize}
	\item Defined as the contact zones where two plates meet.
	\item Many plate boundaries coincide \w\ continental margins, the transition between continental crust and oceanic crust.
\end{itemize}

\section{Location of Earthquakes and Plate Boundaries}\label{sec:location-of-earthquakes-and-plate-boundaries}
\begin{itemize}
	\item The distribution of earthquakes (red dots) is not random.
	\item Most earthquakes occur in narrow seismic belts, which trace the location of plate boundaries.
	The fracturing and sliding that take place along these boundaries as plates move generate the earthquakes.
	\item In contrast, plate interiors remain relatively earthquake-free \bc\ very little movement occurs \win\ them.
\end{itemize}

\section{Three Types of Plate Boundaries}\label{sec:three-types-of-plate-boundaries}
\begin{itemize}
	\item The three types of plate boundaries are distinguished by the nature of relative plate movement.
	\begin{description}
		\item[Divergent boundary] two plates move away from each other.
		\item[Convergent boundary] two plates move towards each other; downgoing plate sinks beneath the overriding plate.
		\item[Transform boundary] two plates slide past each other on a vertical fault surface.
	\end{description}
\end{itemize}

\section{Divergent Boundaries and Seafloor Spreading}\label{sec:divergent-boundaries-and-seafloor-spreading}
\begin{itemize}
	\item Two oceanic plates move apart (diverge) by the process of seafloor spreading at mid-ocean ridges.
	\item As the plates move apart, new ocean floor forms along the divergent boundaries.
	\item Rising asthenosphere melts beneath the ridge axis.
\end{itemize}

\section{Mid-Ocean Range}\label{sec:mid-ocean-range}
\begin{itemize}
	\item All new seafloor forms at mid-ocean ridges, and as it moves away from the ridge axis, it becomes progressively older.
	\item The ridge itself sits much shallower than the surrounding abyssal plains.
	\item Along its length, the ridge is segmented and offset by transform-fault fracture zones.
\end{itemize}

\subsection{Seafloor Age Map}\label{subsec:seafloor-age-map}
\begin{itemize}
	\item The youngest oceanic crust lies along ridge axis.
	\item The oldest oceanic crust is found farthest from the ridge axis, near subduction zones where the seafloor is eventually recycled back into the mantle.
\end{itemize}

\section{Convergent Boundaries and Subduction}\label{sec:convergent-boundaries-and-subduction}
\begin{itemize}
	\item At convergent boundaries, two plates move toward each other.
	\item When at least one of these plates is oceanic, it bends and sinks into the asthenosphere beneath the overriding plate.
	This process is called \emph{subduction}.
	\item \Bc\ of this, convergent boundaries are also known as subduction zones.
	\item Subduction has important implications:
	\begin{itemize}
		\item It consumes old oceanic lithosphere, recycling it back into the mantle.
		\item For this reason, geologists often call convergent boundaries \emph{consuming boundaries}.
		\item At the surface, they appear as long, deep oceanic trenches.
	\end{itemize}
\end{itemize}

\section{Earthquakes at Subduction Zones}\label{sec:earthquakes-at-subduction-zones}
\begin{itemize}
	\item As the downgoing plate descends, it grinds along the base of the overriding plate.
	This friction produces large earthquakes, which often occur relatively close to the Earth's surface near trenches.
	\item Earthquakes also occur \win\ the downgoing plate itself as it bends and sinks into the mantle ($\sim660$ km).
	This distinct band of earthquakes marking this sinking slab is called a Wadati-Benioff zone.
	\item At depths greater than 660 km, plates continue sinking into the mantle.
	However, earthquakes stop occurring \bc\ the slab material adjusts to high pressures and temperatures in a more ductile manner rather than fracturing.
\end{itemize}

\section{Transform Boundaries}\label{sec:transform-boundaries}
\begin{itemize}
	\item Researchers discovered that mid-ocean ridges are not continuous lines but are broken into short segments.
	\item These segments are offset at their ends.
	\item At each offset, a fracture zone develops.
	This fracture zone runs at a right angle to the ridge axis and connects one ridge segment to the next.
	\item At a transform boundary, one plate slides slideways relative to its neighbor along a vertical fault.
	The slip direction is horizontal.
	\item No new plate is created, and no old plate is consumed at a transform boundary.
	\item Fracture zone intersects the end of each ridge segment at a right angle and links it to the next ridge segment.
	\item Only the segment of the fracture zone between two ridge segments is active -- earthquakes happen only along the active transform fault.
\end{itemize}

\section{Special Locations}\label{sec:special-locations}

\subsection{Triple Junctions}\label{subsec:triple-junctions}
\begin{itemize}
	\item A triple junction is a place where three plates intersect.
\end{itemize}

\subsection{Hot Spot Volcanoes}\label{subsec:hot-spot-volcanoes}
\begin{itemize}
	\item Hot spot volcanoes form where a stationary plum of hot mantle rises and melts through a moving tectonic plate, creating a chain of volcanoes
	\begin{itemize}
		\item Active volcano represents the present-day location of the magma source.
		\item The younger and active volcano moves off the hotspot and become inactive, and another, still younger one forms.
		\item Associated chain of inactive volcanic islands is known as a hotspot track, and it provides clues to the direction and rate of plate movement.
	\end{itemize}
	\item Big Island of Hawai'i lies over 4,000 km from the nearest ridge or trench (oceanic hot spot).
	\item Yellowstone National Park lies into the interior of the North American Plate (continental hot spot).
\end{itemize}

\section{Formation of Plate Boundaries -- Rifting}\label{sec:formation-of-plate-boundaries-Rifting}
\begin{itemize}
	\item Most new divergent boundaries form when a continent splits and separates into two continents.
	\item When continental lithosphere stretches and thins, faulting takes place, and volcanoes erupt.
	Eventually, the continent splits in two, and a new ocean basin forms.
	\item Continental rifting is the initial process and seafloor spreading is a later, more advanced stage of this process.
	\item Rift valley in Iceland
	\begin{itemize}
		\item Stretches along the Mid-Atlantic Ridge, where the North American and Eurasian tectonic plates are pulling apart from each other.
		\item The Thingvellir rift grows about one centimeter (0.4 in) each year.
	\end{itemize}
\end{itemize}

\section{Death of Plate Boundaries -- Continental Collision}\label{sec:death-of-plate-boundaries----continental-collision}
\begin{itemize}
	\item A convergent boundary ends when buoyant lithosphere enters a subduction zone.
	\item Subduction continues until the oceanic plate is fully consumed, and then the two continents collide.
	\item During collision:
	\begin{itemize}
		\item The oceanic plate detaches and sinks into the mantle.
		\item Rocks in the collision zone are broken, bent, and compressed, forming large mountain ranges.
		\item The Earth's surface rises, and the crust thickens significantly due to the stacking and deformation of rock layers.
	\end{itemize}
\end{itemize}

\section{Different Geological Settings}\label{sec:different-geological-settings}
\begin{itemize}
	\item Five geological settings related to volcanism
	\begin{itemize}
		\item Island arc (oceanic-oceanic subduction)
		\item Continental arc (oceanic-continental subduction)
		\item Hot spot
		\item Mid-ocean ridge
		\item Rift
	\end{itemize}
\end{itemize}

\section{Velocity of Plate Motions}\label{sec:velocity-of-plate-motions}
\begin{itemize}
	\item The velocity of plate motion can be described in two ways:
	\begin{description}
		\item[Relative plate velocity] describes motion of one plate relative to another.
		\item[Absolute plate velocity] describes motion of one plate compared to a fixed reference plate.
	\end{description}
	\item Estimating absolute plate velocity comes from the assumption that the location of a hot spot does not change much over time, in which case the hot-spot track on a plate provides a record of the plate's absolute velocity.
	\item Plate velocities can now be measured using global positioning system (GPS) satellites.
	\item GPS measurements in southern California show that the region west of the San Andreas fault system, a plate boundary, is moving northwest up to 6 cm/year.
	The length of an error represents the velocity.
\end{itemize}

\section{Changing Face of Earth}\label{sec:changing-face-of-earth}
\begin{itemize}
	\item As a result of plate tectonics, the map of Earth's surface changes slowly and continuously.
	\item The assembly and the later breakup of Pangaea during the past 400 million years.
\end{itemize}
%</Lecture-5>

\end{document}