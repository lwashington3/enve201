%! Author = Len Washington III
%! Date = 9/8/25

% Preamble
\documentclass[
	lecture={6},
	title={Plate Tectonics}
]{enve201notes}

\begin{document}

\setcounter{chapter}{5}
%<*Lecture-6>
\chapter{Plate Tectonics}\label{ch:plate-tectonics}
\section{Consequences of Tectonic Activity}\label{sec:consequences-of-tectonic-activity}
The movement and interaction of tectonic plates lead to several major geological events.
\begin{description}
	\item[Volcano] an opening (vent) where molten rock reaches the Earth's surface.
	Volcanoes can also form mountains built from the products of repeated eruptions.
	\item[Earthquake] episodes of ground shaking caused by the sudden release of energy as plates slip along faults.
	\item[Mountain building] the formation of mountain belts.
	This involves uplift (the crust rising) and deformation (rocks bending, breaking, or flowing) due to stress as compression, tension, or shearing.
\end{description}

\section{Volcanoes}\label{sec:volcanoes}

\begin{itemize}
	\item Beneath a volcano, magma rises through cracks in the crust and collects in a \textbf{magma chamber}.
	\item When pressure builds up, some of this magma erupts at the surface--either through the central vent or along the volcano's flanks.
	\item The style of eruption depends largely on the viscosity of the magma.
	\begin{itemize}
		\item High-viscosity magma
		\item Low-viscosity magma
	\end{itemize}
	\item \Bc\ of this variation, volcanoes come in many different forms.
\end{itemize}

\section{Products of Volcanic Eruptions}\label{sec:products-of-volcanic-eruptions}
\begin{itemize}
	\item Volcanic eruption transfers materials from inside the Earth to our planet's surface.
	\item Products of an eruption come in different forms:
	\begin{description}
		\item[Lava flows] molten rock that moves over the ground.
		\item[Pyroclastic debris] fragments blown out of a volcano.
		\item[Volcanic gases] expelled vapor and aerosols
	\end{description}
\end{itemize}

\subsection{Lava Flows}\label{subsec:lava-flows}
\begin{itemize}
	\item Outpouring of molten rock, or magma, during eruption.
	\item Lava can be thin and runny or thick and sticky depending on viscosity, which is due to composition (especially silica content), temperature, and gas content.
	\item Higher silica $\rightarrow$ higher viscosity $\rightarrow$ less ability to flow.
	\begin{description}
		\item[Basaltic (mafic) lavas] low silica, low viscosity $\rightarrow$ runny, travel long distance
		\item[Ryolitic (felsic) lavas] high silica, high viscosity $\rightarrow$ thick, pile up near the vent
		\item[Andesite lavas] intermediate properties
	\end{description}
\end{itemize}

\subsection{Pyroclastic Debris}\label{subsec:pyroclastic-debris}
Fragmented igneous materials forcefully ejected from a volcano.
\begin{itemize}
	\item Basaltic eruptions produce relatively little pyroclastic debris.
	\item Andesitic and Rhyolitic eruptions are much richer in silica and often generate immense quantities of pyroclastic debris, much more than basaltic eruptions.
\end{itemize}

\subsection{Volcanic Gases}\label{subsec:volcanic-gases}
\begin{itemize}
	\item Most magma contains dissolved gases (volatiles):
	\begin{itemize}
		\item Water vapor (\ce{H2O}), Carbon Dioxide (\ce{CO2}), sulfur dioxide (\ce{SO2}), and hydrogen sulfide (\ce{H2S}).
	\end{itemize}
	\item As magma rises toward the surface, dissolved gases escape \bc:
	\begin{itemize}
		\item Lower pressure near the surface reduce the ability of magma to hold gas.
		\item Crystallization excludes gases (they don't fit into crystals), so gas concentration in the melt increases until bubbles form.
	\end{itemize}
	\item Eruption style:
	\begin{itemize}
		\item Low-viscosity basaltic magma: gases escape easily $\rightarrow$ eruptions are typically gentle
		\item High-viscosity rhyolitic magma: gases are trapped, pressure builds up $\rightarrow$ eruptions are often explosive and violent.
	\end{itemize}
\end{itemize}

\subsection{Other Volcanic Deposits}\label{subsec:other-volcanic-deposits}
\begin{description}
	\item[Pyroclastic deposits] fragments ejected during an eruption that accumulate directly from ash clouds in the atmosphere or from hot avalanches of debris rushing down the volcano's flank.
	\item[Volcanic-sedimentary deposits] volcanic material that has been reworked and redeposited after the eruption--for example, by debris flows or lahars carrying volcanic ash and rock downslope.
	\item[Fragmented lava deposits] angular debris produced when lava breaks apart while flowing on the surface, \wo\ being ejected into the air.
\end{description}

\section{Geological Settings of Volcanism}\label{sec:geological-settings-of-volcanism}
\begin{itemize}
	\item Different styles of volcanism occur at different locations on Earth.
\end{itemize}

\subsection{Mid-Ocean Ridge Submarine Eruptions}\label{subsec:mid-ocean-ridge-submarine-eruptions}
\begin{itemize}
	\item Mid-ocean ridge volcanoes develop along fissures parallel to the ridge axis.
	\item Products of mid-ocean ridge volcanism cover 70\% of Earth's surface.
	\item Mid-ocean ridge volcanoes are not all continuously active.
	\item We don't generally see this volcanic activity, \bc\ the ocean hides most of it beneath a blanket of water.
\end{itemize}

\section{Volcanism of Continental Rifts}\label{sec:volcanism-of-continential-rifts}
\begin{itemize}
	\item Igneous activity of rift zones occurs \bc\ the thinning and stretching of the continental lithosphere reduces the pressure on the underlying asthenosphere.
	\item As the asthenosphere rises to shallower depths, it partially melts, producing magma.
	\item Some of this magma rises directly to the surface, erupting as lava flows or volcanic cones.
	\item A portion of the magma, however, stalls and crystallizes at the base of the crust or \win\ crustal fractures, forming intrusive igneous bodies.
\end{itemize}

\section{Volcanic Arcs at Convergent Boundaries}\label{sec:volcanic-arcs-at-converegent-boundaries}
\begin{itemize}
	\item Most subaerial volcanoes on the Earth lies along convergent boundaries (subduction zones).
	\item Island arc: magma rises from the mantle through the oceanic crust.
	\item Continental arc: volcanoes grow on continental crust.
	\item The \textbf{Ring of Fire} defines the location of most subduction-related volcanoes.
\end{itemize}

\section{Hot-Spot Volcanism}\label{sec:hot-spot-volcanism}
\begin{itemize}
	\item Host spots are volcanic regions fed by mantle plumes.
	Unlike plate boundary volcanism, hot spot volcanism occurs in the middle of tectonic plates.
	\item Oceanic host spots are created when rising mantle plumes undergo decompression melting beneath oceanic lithosphere (e.g., Hawaiian Islands).
	\item Continental hot spots occur when mantle plumes rise beneath continents, causing partial melting of both mantle and continental crust (e.g., Yellowstone).
\end{itemize}

\section{Volcanic Hazards Due to Eruptive Materials}\label{sec:volcanic-hazards-due-to-eruptive-materials}
\begin{itemize}
	\item Threat from lava flows
	\begin{itemize}
		\item Basaltic lava from effusive eruptions is the greatest thread \bc\ it can spread over a broad area.
	\end{itemize}
	\item Pyroclastic debris flows
	\begin{itemize}
		\item Pyroclastic flows can move extremely fast (100-300 km/h) and are so hot (500-1,000\textdegree{}C) that they represent a profound hazard to humans and the environment.
	\end{itemize}
	\item Volcanic Ash and Lapilli
	\begin{itemize}
		\item During a large explosive eruption, ash and lapilli erupt into the air, later fall back to the ground.
		Ashfalls can completely bury landscapes, killing plants and crops.
	\end{itemize}
\end{itemize}

\section{Other Hazards Related to Eruptions}\label{sec:other-hazards-related-to-eruptions}
\subsection{Threat of blast}\label{subsec:threat-of-blast}
\begin{itemize}
	\item When explosions are eject sideways, they can create severe blast hazards.
\end{itemize}

\subsection{Threat of landslide}\label{subsec:threat-of-landslide}
\begin{itemize}
	\item Eruptions commonly trigger large landslides along a volcano's flanks.
	\item Volcanic debris, composed of ash and solidified lava that erupt earlier, can move downslope fast (250 km/h) and far.
\end{itemize}

\subsection{Threat of lahars}\label{subsec:threat-of-lahars}
\begin{itemize}
	\item Mixing volcanic ash and other debris \w\ water produces a lahar, an ash slurry that resembles wet concrete, very thick and dense.
	\item Lahars may develop when heavy rains happen during an eruption, or in regions where snow and ice cover an erupting volcano, for the eruption melts the snow and ice, thereby generating a supply of water.
\end{itemize}

\subsection{Threat of earthquakes}\label{subsec:threat-of-earthquakes}
\begin{itemize}
	\item Earthquakes accompany almost all major volcanic eruptions \bc\ the movement of magma break rocks underground.
\end{itemize}

\subsection{Threat of tsunamis}\label{subsec:threat-of-tsunamis}
\begin{itemize}
	\item Where explosive eruptions occur at an island arc; the blast and the underwater collapse of a caldera can generate huge sea waves or tsunamis, tens of meters high.
\end{itemize}

\subsection{Threat of gas}\label{subsec:threat-of-gas}
\begin{itemize}
	\item Volcanoes erupt not only solid material, but also large quantities of gas such as water vapor (\ce{H2O}), carbon dioxide (\ce{CO2}), sulfur dioxide (\ce{SO2}), and hydrogen sulfide (\ce{H2S}).
	\item Usually, eruption of gases accompanies the eruption of lava and ash.
\end{itemize}

\section{Active, Dormant and Extinct Volcanoes}\label{sec:active-dormant-and-extinct-volcanoes}
\begin{itemize}
	\item The threat from a volcano depends on the likelihood of eruption.
	\item Tectonic processes will eventually shut off volcanoes' magma source, then erosion takes over.
	\begin{description}
		\item[Active] erupting, recently erupted, or likely to erupt
		\item[Dormant] hasn't erupted in hundreds to thousands of years
		\item[Extinct] no longer capable of erupting
	\end{description}
	\item How to determine?
	\begin{itemize}
		\item Examine the historical record
		\item Determine the age of erupted rocks
		\item Search for evidence that the volcano sill lies \win\ a tectonically active area
		\item Examine the landscape character of the volcano (shape)
	\end{itemize}
\end{itemize}

\section{Predicting Eruption}\label{sec:predicting-eruption}
\begin{itemize}
	\item Long-term prediction comes from:
	\begin{itemize}
		\item Recurrence interval - the average time between eruptions
		\item Age of erupted layers making up the volcano
	\end{itemize}
	\item Indicators for volcanic unrest:
	\begin{description}
		\item[Earthquake activity] movement of magma generates vibrations in the Earth.
		\item[Changes in heat flow] the presence of hot magma increases the local heat flow, the amount of heat passing upward through rock.
		\item[Changes in shape] As magma fills the magma chamber inside a volcano, it pushes outward and cna cause the surface of the volcano to bulge.
		\item[Increase in gas and steam emission] Gases bubbling out of the magma and steam formed as the magma heats groundwater percolate upward through cracks in the Earth and rise from the volcanic vent.
	\end{description}
\end{itemize}

\section{Earthquake}\label{sec:earthquake}
The shaking of the Earth's surface resulting from a sudden release of \textbf{energy}, most of which is a consequence of \textbf{plate movement}.
\begin{itemize}
	\item Before an earthquake, rock bends elastically, like a stick arch in your hands.
	\item Eventually, the rock break, and sliding suddenly occurs a fault.
	This break generates vibrations.
	\item When the vibrations produced, the land surface lurched back and forth and bounced up and down - the ground shaking is called an earthquake.
	\item Almost 1 million detectable earthquakes happen every year.
	\item Most cause no damage or causalities, either \bc\ they are too small or \bc\ they occur in unpopulated areas.
	\item A few hundred earthquakes per year rattle the ground sufficiently to crack buildings and injure their occupants.
	\item Every 5 to 20 years, on average, a great earthquake triggers a horrific calamity.
\end{itemize}

\subsection{Causes of the Earthquakes}\label{subsec:causes-of-the-earthquakes}
\begin{itemize}
	\item Seismic activity can result from a variety of geologic and human-induced processing including:
	\begin{itemize}
		\item Sudden formation of a new fault
		\item Sudden slip on an existing fault
		\item Phase change in minerals
		\item Volcanic activity
		\item Giant landslides
		\item Meteorite impacts
		\item Underground nuclear explosions
	\end{itemize}
\end{itemize}

\section{Location of an Earthquake}\label{sec:location-of-an-earthquake}
\begin{itemize}
	\item Hypocenter (focus):
	\begin{itemize}
		\item Actual point inside the Earth where an earthquake begins.
		\item The spot along the fault where rocks first rupture and start slipping, releasing seismic energy.
	\end{itemize}
	\item Epicenter:
	\begin{itemize}
		\item The point directly above the hypocenter, projected up to the Earth's surface.
	\end{itemize}
\end{itemize}

\section{Faults and Related Features}\label{sec:faults-and-related-features}
\begin{itemize}
	\item Fault: a fracture or break in the Earth's crust along which rocks or sediments have moved relative to each other.
	\begin{itemize}
		\item Blind faults: Some faults do not extend to the Earth's surface while they are active.
		\item When a fault does intersect the surface, it can offset the ground.
		This offset creates a fault scarp.
		\item Fault line (fault trace): visible line where the fault plane meets the Earth's surface.
	\end{itemize}
	\item Movement along faults:
	\begin{itemize}
		\item Miners described fault motion observing the relative movement of the hanging wall compared to the footwall.
	\end{itemize}
\end{itemize}

\subsection{Basic Types of Faults}\label{subsec:basic-types-of-faults}
\subsubsection{Normal Fault}{subsubsec:normal-fault}
\begin{itemize}
	\item Hanging wall moves down relative to footwall.
	\item Faults form during extension (stretching) of the crust.
\end{itemize}

\subsubsection{Reverse fault (steep) or a thrust fault (shallow)}{subsubsec:reverse-fault-(steep)-or-a-thrust-fault-(shallow)}
\begin{itemize}
	\item Hanging wall moves up relative to footwall.
	\item Faults form during compression (squeezing or shortening) of the crust.
\end{itemize}

\subsubsection{Strike-slip fault}{subsec:strike-slip-fault}
\begin{itemize}
	\item Blocks slide past each other horizontally.
	\item No vertical displacement takes place.
\end{itemize}

\subsection{Development of New Faults}\label{subsec:development-of-new-faults}
\begin{itemize}
	\item Faults form when tectonic forces add stress (push, pull, or shear) to rock.
	Think of a block of rock held by clamps:
	\begin{itemize}
		\item When force is applied, the rock bends slightly at first \wo\ breaking.
		\item \W\ continued stress, small cracks begin to form inside the rock.
		These cracks slowly grow and start to connect \w\ each other.
		\item Eventually, the connected cracks create a continuous fracture that cut across the entire block of rock.
	\end{itemize}
\end{itemize}

\subsection{Foreshocks, Mainshock, Aftershocks}\label{subsec:foreshocks-mainshock-aftershocks}
\subsubsection{Foreshocks}{subsubsec:foreshocks}
\begin{itemize}
	\item Sometimes, smaller earthquakes occur before the mainshock.
	\item They happen as minor cracks develop.
\end{itemize}

\subsubsection{Mainshock}{subsubsec:mainshock}
The largest and most powerful earthquake in a sequence.
It usually releases the bulk of accumulated stress along a fault.

\subsubsection{Aftershocks}{subsubsec:aftershocks}
\begin{itemize}
	\item After the mainshock, a series of small quakes called aftershocks often follow.
	\item These occur \bc\ the fault system doesn't settle into a stable configuration immediately after the main rupture.
	\item Aftershocks may continue for weeks, months, or even years.
\end{itemize}

\subsection{Seismic Waves}\label{subsec:seismic-waves}
\begin{itemize}
	\item Earthquake energy moves through rock and sediment in the form of vibration, and this movement is called seismic waves, or earthquake waves.
	\item Different types of seismic waves travel at different velocities.
	\begin{description}
		\item[P-waves] travel by compressing and expanding the material parallel to the wave-travel direction.
		P-waves are the fastest seismic waves, and they travel through solids, liquids, and gases.
		\item[S-waves] travel by moving material back and forth, perpendicular to the wave-travel direction.
		S-waves are slower than P-waves, and they travel only through solids, never liquids or gases.
		\item[Surface waves] travel along Earth's exterior.
		Surface waves are the slowest and most destructive.
	\end{description}
\end{itemize}

\subsection{Record of Earthquakes}\label{subsec:record-of-earthquakes}
\definition{Sesimographs}{instruments that record ground motion.}
\begin{itemize}
	\item A weighted pen on a spring traces movement of the frame.
	\item Vertical motion is recorded as up-and-down movement; horizontal motion is recorded as back-and-forth motion.
\end{itemize}

\noindent\definition{Seismogram}{the data record from a seismograph.}
\begin{itemize}
	\item Seismogram depicts earthquake wave behavior, particularly the arrival times of the different waves, which are used to determine the distance to the epicenter.
\end{itemize}
%
\begin{itemize}
	\item Seismic waves arrive at a station in sequence.
	\begin{itemize}
		\item P-waves are first (fastest).
		\item S-waves are second (slower).
		\item Surface waves are last.
	\end{itemize}
	\item Measure the time difference between the arrival of P-waves and S-waves.
	\item \Bc\ P-waves and S-waves travel at different velocities through the Earth, this time gap increases \w\ distance from the epicenter.
	\item Locating an Earthquake epicenter:
	\begin{itemize}
		\item Use travel-time curves for P- and S-waves to find out how far away an earthquake occurred.
		\item The S--P interval gives the distance from the seismograph station to the epicenter.
		\item On a map, draw a circle around each seismograph station \w\ a radius equal to that calculated distance.
		\item The point where the circles overlap marks the earthquake's epicenter.
	\end{itemize}
\end{itemize}

\subsection{Size of Earthquake}\label{subsec:size-of-earthquake}
\begin{itemize}
	\item Seismologists have developed two scales to define size:
	\begin{description}
		\item[Mercalli intensity scale] depends on human perception of ground shaking and the damage resulting from it at given locality.
		\item[Magnitude scale] focuses on the measured amount of ground motion, as recorded by a seismograph at a specified distance from the epicenter.
		\begin{itemize}
			\item Richter scale
			\item \definition{Moment magnitude scale}{provide the most accuracte representation of an earthquake size (most common).}
		\end{itemize}
	\end{description}
\end{itemize}

\subsection{Where Do Earthquakes Occur?}\label{subsec:where-do-earthquakes-occur?}
\begin{itemize}
	\item Shallow earthquakes happen in the top 60 km of the Earth.
	\item Intermediate earthquakes take place between 60 and 330 km.
	\item Deep earthquakes occur down to a depth of $\sim660$ km.
	\item Shallow earthquakes occur at divergent and transform boundaries.
	\item Intermediate and deep earthquakes occur at convergent boundaries.
\end{itemize}

\subsection{Earthquakes at Plate Boundaries}\label{subsec:earthquakes-at-plate-boundaries}
\begin{itemize}
	\item Divergent boundary (mid-ocean ridges)
	\begin{itemize}
		\item Two oceanic plates form and move apart.
		\item Divergent boundary consists of spreading segments linked by transform faults.
	\end{itemize}
	\item Two types of faults develop at divergent boundaries:
	\begin{itemize}
		\item \emph{Normal faults} at the spreading ridge axis;
		\item \emph{Strike-slip faults} along the transforms.
	\end{itemize}
	\item Earthquakes along mid-ocean ridges have foci at depths of less than 25 km, and classified as shallow earthquakes.
	\item Mid-ocean ridge earthquakes don't cause damage.
\end{itemize}

\subsubsection{Transform-Plate Boundary}{subsubsec:transform-plate-boundary}
\begin{itemize}
	\item Most are strike-slip faults.
	\item All transform-fault earthquakes have a shallow focus, so the larger earthquakes on land cause immense damage.
	\item The San Andreas fault cuts through western California where the Pacific plate shears north and the North American plate south.
	\item The San Francisco earthquake of 1906 ($\text{M}_{\text{W}}$ of 7.9) serves an example of a continental transform-fault earthquake.
\end{itemize}

\subsubsection{Convergent Boundary}{subsubsec:convergent-boundary}
\begin{itemize}
	\item One plate subducts under another, and several different kinds of earthquakes take place (shallow, intermediate, and deep earthquakes):
	\begin{description}
		\item[Normal faults] form where the downgoing slab bends, seaward of trench.
		\item[Large thrust faults] occur at the contact between downgoing and overriding plates.
		Shear on the faults can produce disastrous, shallow earthquakes.
	\end{description}
	\item Convergent boundaries host shallow, intermediate and deep earthquakes.
	\item The earthquakes occur in downgoing slab as it sinks into the mantle, and a sloping band of seismicity called a \textit{Wadati Benioff zone}.
	\item Earthquakes are rare below 660 km as the mantle becomes too ductile.
\end{itemize}


\subsection{Earthquakes Due to Continental Rifting}\label{subsec:earthquakes-due-to-continental-rifting}
\begin{itemize}
	\item The stretching of continental crust at continental rifts generates normal faults.
	\item Shallow earthquakes rattle the landscape.
	\item In contrast to mid-ocean ridge earthquakes, earthquakes in rifts occur on land and can be located under or near populated areas.
	\item Earthquakes in Africa occur mostly along the East African rift.
\end{itemize}

\subsection{Earthquakes Due to Collision}\label{subsec:earthquakes-due-to-collision}
\begin{itemize}
	\item Two continents collide when the oceanic lithosphere that once separated them has been completely subducted.
	\item Such collision produce great mountains.
	\item Earthquakes in Southern Asia occur primarily in crust deforming due to the collision of Eurasia and India.
\end{itemize}

\subsection{Intraplate Earthquakes}\label{subsec:intraplate-earthquakes}
\begin{itemize}
	\item Some earthquakes (about 5\%) affect the interiors of plates and are not associated \w\ late boundaries, active rifts, or collision zones.
	\item Intraplate earthquakes are caused by the stress applied to continental lithosphere triggers lip on pre-existing faults in the crust -- long-lived weak ``scars'' in the crust.
\end{itemize}

\subsection{Earthquake Waves}\label{subsec:earthquake-waves}
Earthquake Waves arrive in a distinct sequence \w\ different motions.
\begin{description}
	\item[P-waves] are first to arrive.
	They produce a rapid, bucking, up-and-down motion.
	\item[S-waves] arrive next (second).
	They produce a pronounced back-and-forth motion.
	This motion is much stronger than that from P-waves.
	S-waves cause extensive damage.
\end{description}

\noindent Surface waves are delayed traveling along the exterior.
\begin{description}
	\item[L-waves] follow quickly behind the S-waves.
	They cause the ground to writhe like a snake.
	\item[R-waves] are the last to arrive.
	The land surface undulates like ripples across a pond.
	These waves usually last longer than the other kinds.
	R-waves cause extensive damage.
\end{description}

\subsection{Earthquake Damages Due to Vibration}\label{subsec:earthquake-damages-due-to-vibration}
\begin{itemize}
	\item Building floors ``pancake''.
	\item Bridges and roadways topple.
	\item Bridge support crush.
	\item Masonry walls break apart.
\end{itemize}

\subsection{Earthquake Hazards}\label{subsec:earthquake-hazards}
\subsubsection{Landslides}{subsubsec:earthquake-hazards-landslides}
\begin{itemize}
	\item A landslide is the downslope tumbling, sliding, or flow of soil, rock or debris under the influence of gravity.
	\item Strong shaking can destabilize steep slopes or ground made up of loose or weak sediment, causing them to suddenly give way.
	\item Landslides frequently accompany earthquakes in places \w\ topographic relief (hills, mountains, or steep valleys).
\end{itemize}

\subsubsection{Liquefaction}{subsubsec:earthquake-hazards-liquefaction}
\begin{itemize}
	\item During strong shaking, seismic waves increase the pressure of water in the pore spaces of saturated sediments.
	As pore-water pressure rises, friction between grains is reduced.
	\item The sediment temporarily loses its strength and behave like a fluid (slurry).
	\begin{itemize}
		\item Sandy soil can behave like quicksand.
		\item Clay-rich soils can transform into quickclay.
	\end{itemize}
	\item Liquefaction causes soil to lose strength.
	Land, and the structures on it, will slump and flow.
	Buildings may founder and topple over intact.
\end{itemize}

\subsubsection{Fire}{subsubsec:earthquake-hazards-fire}
\begin{itemize}
	\item Fire is a common result of earthquakes.
	\item The shaking during an earthquake can tip over lamps, stoves, or candles \w\ open flames, and it may break wires or topple power lines, generating sparks.
	\item Firefighters are often powerless to combat fire \wo\ road access, no water, and too many hot spots.
	\item Fire may greatly magnify the destruction and tool in human lives.
\end{itemize}

\subsubsection{Tsunamis}{subsubsec:earthquake-hazards-tsunamis}
\begin{itemize}
	\item Tsunami means harbor wave in Japanese.
	\item Tsunamis result from displacement of the sea floor by an earthquake, submarine landslide, or volcanic explosion that displaces the entire volume of overlying water.
\end{itemize}

\subsection{Earthquake Prediction}\label{subsec:earthquake-prediction}
\begin{itemize}
	\item Can we predict earthquakes?
	Yes and no.
	\item We \emph{can} predict in the long term (tens to thousands of years).
	\item We \emph{cannot} predict in the short term (hours and weeks).
	\item Hazards can be mapped to assess risk and develop building codes, implement land-use planning, and disaster response.
	\item Earthquakes have precursors:
	\begin{itemize}
		\item Clustered foreshocks
		\item Crustal strain
		\item Level changes in wells
		\item Gases (\ce{Rn}, \ce{He}) in wells
		\item Unusual animal behavior
	\end{itemize}
\end{itemize}

\subsection{Preparedness}\label{subsec:earthquake-preparedness}
\begin{itemize}
	\item Map active faults and areas likely to liquefy from shaking.
	\item Develop construction codes to reduce building failures.
	\item Regulate land use to control development in hazard areas.
\end{itemize}
%</Lecture-6>

\end{document}