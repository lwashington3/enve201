%! Author = Len Washington III
%! Date = 9/8/25

% Preamble
\documentclass[
	lecture={7},
	title={Minerals}
]{enve201notes}

% Packages

% Document
\begin{document}

\setcounter{chapter}{6}
%<*Lecture-7>
\chapter{Minerals}\label{ch:minerals}

\section{Minerals}\label{sec:minerals}
\begin{itemize}
	\item Minerals have a specialized geologic definition: naturally occurring, (mostly) inorganic crystalline solids formed by geologic processes and \w\ a definite chemical composition.
	\item Minerals make up most rocks and sediments of the solid Earth.
	\item Minerals used as resources by mankind:
	\begin{itemize}
		\item Industrial minerals provide the raw materials for manufacturing chemicals, concrete, and wallboard.
		\item Ore minerals are the source of valuable metals, like copper and gold (e.g., Malachite is a type of copper ore (\ce{Cu2[CO3][OH]2}); it contains copper plus other chemicals
		\item Gems delight the eye in jewelery.
	\end{itemize}
	\item Certain minerals pose health and environmental hazards.
\end{itemize}

\section{Mineral vs Glass}\label{sec:mineral-vs-glass}
\begin{itemize}
	\item Both mineral and glass are solids, but minerals are crystalline, while glass is not.
	\begin{itemize}
		\item Atoms, ions, or molecules in a \emph{mineral} are ordered into a geometric arrangement.
		\item Atoms, ions, or molecules in \emph{glass} are arranged in a semi-chaotic way, in small clusters or chains that are neither oriented in the same way nor spaced at regular intervals.
	\end{itemize}
\end{itemize}

\section{Crystals}\label{sec:crystals}
\begin{itemize}
	\item A crystal is a single, continuous (uninterrupted) piece of a crystalline material bounded by flat crystal surfaces, called crystal faces, that form naturally as the crystal grows.
	\item The angle between two adjacent crystal faces of any mineral specimen is identical to the angle between the corresponding faces of any other specimen of the same mineral.
	\begin{itemize}
		\item e.g., the angle between the faces of the columnar part of a quartz crystal is exactly 120\textdegree{}.
	\end{itemize}
	\item \emph{All minerals are crystalline, but not all crystals are minerals!}
	\item Crystals come in a variety of shapes, including cubes, trapezoids, pyramids, octahedrons, blades, needles, columns, and obelisks.
	\item The geometry of the arrangement defines the crystal structure.
\end{itemize}

\section{Inside a Mineral}\label{sec:inside-a-mineral}
\begin{itemize}
	\item X-ray diffraction (XRD) is still used today to identify minerals!
	\item When an X-ray beam passes through a crystal, it interacts \w\ the regularly spaced atoms inside.
	\item \Bc\ the spacing between atomic planes is similar to the wavelength of X-rays, the beam is diffracted (bent and scattered).
	\item This diffraction produces a distinctive pattern of spots or rings on a detector screen.
\end{itemize}

\section{Crystal's Structure}\label{sec:crystal's-structure}
\begin{itemize}
	\item The geometry of the atomic packing and the nature of chemical bonding determine the mineral's properties.
	\begin{itemize}
		\item The way elements are packed into a mineral crystal lattice depends upon the size and the charge of the ions of that element.
		\item A large central cation requires a large number of anions; a smaller central cation, fewer anions.
		\item e.g., sodium (\ce{Na+}) and chloride (\ce{Cl-}) ions are bonded in a cubic lattice by ionic bonds to form the mineral halite (\ce{NaCl}), commonly known as salt.
	\end{itemize}
\end{itemize}

\subsection{Polymorphs}\label{subsec:polymorphs}
\begin{itemize}
	\item Minerals that share the same chemical composition but have different crystal structures, meaning their atoms are arranged differently.
	\item Diamond vs. Graphite (both Carbon)
	\begin{itemize}
		\item In diamond, each carbon atom bonds \w\ four neighbors in a tight tetrahedral structure, making diamonds extremely hard.
		\item In graphite, carbon atoms form flat sheets arranged in hexagons.
		The sheets are held together by weak bonds, so they easily slide past each other--this is why graphite is soft and works as ``lead'' in pencils.
	\end{itemize}
\end{itemize}

\section{Formation of Minerals}\label{sec:formation-of-minerals}
A new mineral crystal can form in one of five ways:
\begin{description}
	\item[Solidification] of a melt
	\item[Precipitation] from a water solution - atoms, molecules, or ions dissolved in water bond together and separate out from the water.
	\item[Precipitation] from a gas - typically occur around volcanic vents or geysers.
	\item[Solid-state diffusion] - atoms migrate through the crystal and new minerals grow inside the rock.
	\item[Biomineralization] - minerals can form at interfaces between the physical and biological components of the earth system.
\end{description}

\section{Growth of Crystals}\label{sec:growth-of-crystals}
\begin{itemize}
	\item The first step in forming a crystal happens when the chance assembly of a seed, an extremely small crystal, takes place.
	\item Once the seed exists, other atoms in the surrounding material attach themselves to the faces of the seed, and the seed grows into a crystal.
	\item As the crystal grows, its faces move outward but maintain the same orientation.
	\item A growing crystal develops its particular crystal shape based on the geometry of its internal crystal structure.
\end{itemize}

\section{Destruction of Minerals}\label{sec:destruction-of-minerals}
A mineral can be destroyed by melting, by dissolution, or by some other chemical reactions:
\begin{description}
	\item[Melting] involves heating a mineral to a temperature at which thermal vibrations of the atoms or ions in the crystal structure break the chemical bonds holding them to the lattice.
	\item[Dissolution] takes place when a mineral is immersed in a solvent, such as water.
	\item[Chemical reactions] can destroy a mineral when it comes in contacts \w\ reactive materials.
\end{description}

\section{Mineral Identification}\label{sec:mineral-identification}
Mineral identification requires learning mineral physical properties.

\subsection{Color}\label{subsec:mineral-identification-color}
\begin{itemize}
	\item The color or a mineral results from how it interacts \w\ light.
	\item Minerals absorb specific wavelengths of light, and the color we see is the combination of wavelengths that are not absorbed.
	\item Small amounts of chemical impurities can change the color dramatically.
	For example, trace amounts of iron can make quartz appear purple (amethyst) or yellow (citrine).
\end{itemize}

\subsection{Streak}\label{subsec:minearl-identification-streak}
\begin{itemize}
	\item The streak of a mineral is the color of its powdered form, produced by scraping the mineral across an unglazed ceramic plate.
	\item The streak color can be the same or different from the minerals' outward appearance.
	\item Streak color is often more consistent and reliable than external color, which may vary due to impurities or surface weathering.
	\begin{itemize}
		\item e.g., Calcite always leaves a white streak, even though whole crystals may appear white, pink, or clear.
	\end{itemize}
	\item Limitations: the streak test works best for minerals \w\ a hardness of $<6$ on the Mohs scale (so that it can be ground/powdered against the porcelain plate).
\end{itemize}

\subsection{Luster}\label{subsec:minearl-identification-luster}
\begin{itemize}
	\item Luster refers to the way a mineral surface scatters light.
	\item The two main subdivisions of luster are metallic and nonmetallic:
	\begin{description}
		\item[Metallic luster]: minerals that look like metal
		\item[Nonmetallic luster]: minerals that do not look like metal
	\end{description}
\end{itemize}

\subsection{Hardness}\label{subsec:minearl-identification-hardness}
\begin{itemize}
	\item Hardness indicates a mineral's ability to resist scratching, which depends on the strength of the bonds in its crystal structure.
\end{itemize}
\begin{description}
	\item[Relative hardness]: hard minerals can scratch softer ones, but soft minerals cannot scratch harder ones.
	\item[Mohs hardness scale]: Friedrick Mohs arranged common materials in order of relative hardness.
	\begin{itemize}
		\item e.g. If your fingernail (hardness $\sim$ 2.5) scratches a mineral, that mineral must be softer than 2.5.
	\end{itemize}
\end{description}

\subsection{Crystal Habit}\label{subsec:mineral-identification-crystal-habit}
\begin{itemize}
	\item Crystal habit describes the external shape of a well-formed crystal or the overall appearance of a group of crystals that grew together.
	\item Habit reflects the internal atomic arrangement of the mineral.
	\item Common descriptive terms include cubic, prismatic, bladed, platy, and fibrous.
	\begin{itemize}
		\item Wulfentie commonly forms thin, tabular plates.
		\item Kyanite typically forms bladed crystals.
	\end{itemize}
\end{itemize}

\subsection{Cleavage}\label{subsec:mineral-identification-cleavage}
\begin{itemize}
	\item Cleavage is the tendency of a mineral to break along specific planar surfaces that correspond to zones of weaker atomic bonding \win\ the crystal lattice.
	\item The number of cleavage planes and the angles between them are diagnostic features used in mineral identification.
	\item Cleavage can be distinguished from crystal faces \bc\ cleavage repeats throughout the mineral, whereas crystal faces are just the external growth surfaces of the crystal.
\end{itemize}

\subsection{Fracture}\label{subsec:mineral-identification-fracture}
\begin{itemize}
	\item Minerals that lack cleavage break by fracture, which may appear irregular or uneven.
	\item Conchoidal fracture is a special type of fracture that produces smoothly curving, clamshell-shaped surfaces.
	This is commonly seen in glass, quartz, and obsidian.
	\item Fracture is distinct from cleavage \bc\ it does not follow any specific atomic planes \win\ the crystal structure.
\end{itemize}

\subsection{Others}\label{subsec:mineral-identification-others}
Other less common physical properties that are useful for identifying minerals:
\begin{description}
	\item[Effervescence] reactivity \w\ acid
	\item[Magnetism] magnetic attraction
	\item[Taste \& smell]
	\item[Feel] tactile response
	\item[Elasticity] response to bending
	\item[Diaphaneity] relative transparency
	\item[Pizoelectricity] electric charge when squeezes
	\item[Pyroelectricity] electric charge when heated
	\item[Refractive Index] degree of bending light
	\item[Malleability] ability to be pounded into thin sheets
	\item[Ductility] ability to be drawn into this wires
	\item[Sectility] ability to be shaved \w\ a knife
	\item[Specific Gravity]
\end{description}

\section{Hazardous Minerals}\label{sec:hazardous-minerals}
Some minerals contain toxic chemicals that can pose serious health risks.
\begin{itemize}
	\item Arsenopyrite contains arsenic, which can release poisonous compounds when weathered.
	\item Asbestos is a fibrous silicate mineral; when inhaled, its needle-like fibers can lodge in human lungs, causing diseases such as asbestosis, mesothelioma, and lung cancer.
	\item Pulverized silicates (like quartz dust) can cause silicosis, a chronic lung disease, when inhaled over long periods.
\end{itemize}

\section{Mineral Classification}\label{sec:mineral-classification}
Mineralogists distinguish \textbf{seven principal classes} of minerals:
\begin{description}
	\item[Silicates] all silicates contain the \ce{SiO4^{4-}} anionic group.
	\item[Sulfides] consists of a metal cation bonded to a sulfided anion (\ce{S2-}).
	\item[Oxides] consist of metal cations bonded to oxygen atoms.
	\item[Halides] the anion in a halide is a halogen or salt producing ion (such as chloride (\ce{Cl-}) or fluoride (\ce{F-})).
	\item[Carbonates] in carbonate minerals, the molecule \ce{CO3^{2-}} serves as the anionic group.
	\item[Native metals] consist of pure masses of a single metal.
	\item[Sulfates] consist of metal cations bonded to \ce{SO4^{2-}}.
\end{description}

\subsection{Silicates}\label{subsec:silicates}
\begin{itemize}
	\item Silicates are the major rock-forming minerals, making up over 95\% of the continental crust.
	\item All silicates contain the \ce{SiO4^{4-}} anionic group, also called the silica tetrahedron.
	In this structure, four oxygen atoms surrounds a single silicon atom, forming a pyramid-like shape \w\ four triangular faces.
	\item Silicate minerals are grouped based on how these silica tetrahedra are arranged and linked together.
	\item The arrangement depends on how many oxygen atoms are shared between tetrahedra, which in turn determines the silicon-to-oxygen ratio in the mineral.
	\begin{itemize}
		\item Isolated tetrahedra (independent)
		\item Single-chain silicates
		\item Double-chain silicates
		\item Sheet silicates
		\item Framework silicate
	\end{itemize}
\end{itemize}

\subsection{Gemstones and Gems}\label{subsec:gemstones-and-gems}
\begin{itemize}
	\item A gemstone is a mineral that holds special value \bc\ it is rare and considered beautiful.
	\item Gemstones can form in several ways, including solidification from melt, diffusion, precipitation, or chemical interaction between rock and water near Earth's surface.
	\item When a gemstone is cut and polished, it becomes a gem, ready to be set into jewelry.
	\begin{itemize}
		\item Precious stones include diamond, ruby, sapphire, and emerald
		\item Semiprecious stones include topaz, tourmaline, aquamarine, and garnet
		\item Most gemstones are transparent crystals, often showing vibrant colors that add to their appeal.
	\end{itemize}
\end{itemize}

\begin{table}[H]
	\centering
	\caption{Precious and Semiprecious Materals Used in Jewelry}
	\label{tab:precious-materials}
	\begin{tabular}{|p{0.07\textwidth} | p{0.15\textwidth} | p{0.78\textwidth}|}
		\toprule
		Gem Name & Material/Formula & Comments\\
		\midrule
		Amber & Fossilized tree sap & Composed of organic chemicals; amber is not strictly a mineral.\\
		\midrule
		Amethyst & Quartz/\ce{SiO2} & The best examples precipitate from water in openings in igneous rocks; a deep purple version of quartz.\\
		\midrule
		Aquamarine & Beryl/\ce{Be3Al2Si6O18} & A bluish version of emerald.\\
		\midrule
		Diamond & Diamond/\ce{C} & Brought to the surface from the mantle in igneous bodies called diamond pipes; may later be mixed in deposits of sediment.\\
		\midrule
		Emerald & Beryl/\ce{Be3Al2Si6O18} & Occurs in coarse igneous rocks (pegmatites)\\ % TODO: If pegmatites show up in lecture 6, link it
		\midrule
		Garnet & \ce{} &\\
		\midrule
		Jade & \ce{} & \\
		\midrule
		Opal & \ce{} & \\
		\midrule
		Pearl & \ce{} & \\
		\midrule
		Ruby & \ce{} & \\
		\midrule
		Sapphire & \ce{} & \\
		\midrule
		Topaz & \ce{} & \\
		\midrule
		Tourmaline & \ce{} & \\
		\midrule
		Turquoise & \ce{} & \\
		\bottomrule
	\end{tabular}
\end{table}


%</Lecture-7>
\end{document}