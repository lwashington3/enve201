%! Author = Len Washington III
%! Date = 9/8/25

% Preamble
\documentclass[
	lecture={3},
	title={Continental Drift}
]{enve201notes}

% Packages

% Document
\begin{document}

\setcounter{chapter}{3}
%<*Lecture-4>
\chapter{Continental Drift}\label{ch:continental-drift}
\section{Continental Drift}\label{sec:continental-drift}
\begin{itemize}
	\item Alfred Wegener, a German meteorologist, proposed that the continents had once fit together like pieces of a giant jigsaw puzzle, making one vast supercontinent named \emph{Pangaea}.
	\item The phenomenon that Wegener proposed came to be known as \emph{continental drift}.
	\item Wegener presented observations to support the theory of continental drift:
	\begin{itemize}
		\item The fit of the continents
		\item Locations of past glaciations
		\item The distribution of climate belts
		\item The distribution of fossils
		\item Matching geologic units
	\end{itemize}
\end{itemize}

\subsection{Evidence 1 - The Fit of the Continents}\label{subsec:evidence-1-the-fit-of-the-continents}
\begin{itemize}
	\item North America, South America, Africa, and Europe appear to fit together.
	\item When all continents are joined, they form a single supercontinent called Pangaea, \w\ remarkably few overlaps or gaps.
	\item Wegener concluded that the continents once did fit together in the geologic past.
	\item Later, Edward Bullard used a computerized reconstruction to align the continental shelves.
	His results showed how although the fit is not perfect, the mismatches are minimal and the overall alignment is striking.
\end{itemize}

\subsection{Evidence 2 - Locations of Past Glaciations}\label{subsec:evidence-2-locations-of-past-glaciations}
\begin{itemize}
	\item Flowing glacier transport sediments of all sizes, and hard grains carve scratches, called \emph{striations}, into the underlying rock.
	\item The distribution of the glacial deposits (\emph{till}) and the orientation of the striations were mapped. % continuous ice sheet
	\item All late Paleozoic glaciated regions align adjacent to one another on this map, forming a single continuous ice sheet.
\end{itemize}

\subsection{Evidence 3 - The Distribution of Climate Belts}\label{subsec:evidence-3-the-distribution-of-climate-belts}
\begin{itemize}
	\item Paleozoic sedimentary rocks record clues to past climate at the time of sediment deposition.
	\begin{itemize}
		\item In tropical swamps and jungle regions $\rightarrow$ thick accumulation of plant material, which later transform into coal.
		\item In subtropical regions $\rightarrow$ extensive salt deposits and the growth of large sand dunes.
	\end{itemize}
	\item \emph{Wegener's observation}:
	\begin{itemize}
		\item In the equatorial Pangaea belt, late Paleozoic rocks contain abundant coal seams and the relics of reefs.
		\item In the subtropical Pangaea belt, rock layers preserve desert dunes and salt layers.
	\end{itemize}
	\item On today's map, these deposits appear scattered across different continents and latitudes.
\end{itemize}

\subsection{Evidence 4 - The Distribution of Fossils}\label{subsec:evidence-4-the-distribution-of-fossils}
\begin{itemize}
	\item Different continents host different species, since land-dwelling or coastal organisms cannot cross vast oceans.
	Overtime, these species evolve independently on separate continents.
	\item \emph{Wegener's observation}:
	\begin{itemize}
		\item He mapped fossil occurrences of late Paleozoic and early Mesozoic land-dwelling species
		\item The distribution shows that identical fossils appear on continents that were once joined in Pangaea.
	\end{itemize}
\end{itemize}

\subsection{Evidence 5 - Matching Geologic Units}\label{subsec:evidence-5-matching-geologic-units}
\begin{itemize}
	\item Observation if the Atlantic Ocean did not exist:
	\begin{itemize}
		\item Distinctive belts of rock in South America would align seamlessly \w\ those in Africa.
		\item Paleozoic mountain belts on both coasts would connect to form continuous ranges.
	\end{itemize}
	\item A modern construction illustrates:
	\begin{itemize}
		\item The geologic units and mountain belts line up almost perfectly.
		\item The outlines of today's continents drawn in white for reference.
	\end{itemize}
\end{itemize}

\subsection{Criticism of Wegener's Ideas}\label{subsec:criticism-of-wegener's-ideas}
\begin{itemize}
	\item Although Wegener presented strong observational evidence for continental drift, he could not explain the \emph{mechanism}:
	``What force could possibly be great enough to move the immense mass of a continent?''
	\item As a result, most geologists of his time \emph{rejected} the theory of continental drift.
	\item In the three decades that followed Wegener's death, geologists discovered the existence of huge convection cells in the Earth's mantle.
	These convection currents transport hot rock slowly upward from the deep mantle up to the base of the crust.
	\item \emph{Continental movement, seafloor spreading}, and \emph{subduction} are now explained by the concept of plate tectonics.
\end{itemize}

\section{Seafloor Mapping}\label{sec:seafloor-mapping}
\subsection[Depth Measurements]{Scattered Soundings (Depth Measurements) Before World War II}\label{subsec:seafloor-mapping-depth-measurements}
\begin{itemize}
	\item Depth was measured by lowering a cable \w\ a lead weight.
	When the weight hit the seafloor, the length of the cable would indicate the depth.
	\item Each sounding could take hours, so measurements were sparse.
\end{itemize}

\subsection[Sonar]{Sonar After World War II}\label{subsec:sonar}
\begin{itemize}
	\item Ships emitted sound pulses that traveled to the seafloor and returned as echoes.
	\item With the known speed of sound in water: distance = velocity $\times$ time.
	\item Sonar allowed rapid, continuous depth recording, producing detailed seafloor profiles.
\end{itemize}

\section{Bathymetric Map of the Seafloor}\label{sec:bathymetric-map-of-the-seafloor}
\begin{itemize}
	\item \emph{Bathymetric profile} (yellow line, $Y-Y'$) is a graph of depth versus location along a line.
	\item By sailing back and forth across the ocean and collecting many such profiles at different locations, investigators compiled enough data to construct a \emph{bathymetric map} of the seafloor.
\end{itemize}

\section{Seafloor Sediments and Heat Flow}\label{sec:seafloor-sediments-and-heat-flow}
New observations on the oceanic crust:
\begin{description}
	\item[Sediments thicken and age away from ridges]: A layer of clay and microscopic shells becomes progressively thicker and older \w\ increasing distance from the mid-ocean ridge axis, showing that ridges are younger than the deeper ocean floor.
	\item[Heat flow is highest at ridges]: Heat rising from Earth's interior varies across the seafloor, \w\ greater heat flow beneath mid-ocean ridges.
	This suggests molten rock is rising into the crust at ridge axes, carrying heat upward.
\end{description}
\begin{itemize}
	\item Harry Hess' concept of \emph{seafloor spreading}:
	\begin{itemize}
		\item Hot mantle rises beneath mid-ocean ridges and melts.
		\item At the ridge axis, this melts solidifies to create new oceanic crusts.
		\item Once formed, the crust cracks, splits apart, and gradually moves away from the ridge.
		\item This process explained \emph{the creation of new seafloor}.
	\end{itemize}
\end{itemize}

\section{Seafloor Subduction}\label{sec:seafloor-subduction}
\begin{itemize}
	\item Geologists realized that if new ocean floor forms at mid-ocean ridges, the old ocean floor must be consumed elsewhere:
	\begin{itemize}
		\item Deep-sea trenches are the site where oceanic crust bends and sinks back into the mantle.
		\item Frequent earthquakes along trenches confirm this downward movement.
		\item \emph{Subduction} is the process by which oceanic floor descends into the Earth's interior at trenches, balancing seafloor spreading.
	\end{itemize}
\end{itemize}

\section{Locations of Earthquakes}\label{sec:locations-of-earthquakes}
\begin{itemize}
	\item Earthquakes occur in distinct seismic belts, not scattered randomly across the globe.
	\item Their distribution in ocean basins traces the outlines of mid-ocean ridges, transform faults, and deep-sea trenches.
\end{itemize}

%</Lecture-4>
\end{document}