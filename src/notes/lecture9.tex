%! Author = Len Washington III
%! Date = 9/22/25

% Preamble
\documentclass[
	lecture={9},
	title={Rocks -- Igneous}
]{enve201notes}

% Packages

% Document
\begin{document}

\setcounter{chapter}{8}
%<*Lecture-9>
\chapter{Rocks}\label{ch:rocks}

\section{Why Study Rocks?}\label{sec:why-study-rocks?}
\begin{itemize}
	\item Foundation for geology: Understanding rocks in essential for interpreting many geologic processes such as volcanic eruptions, mountain building, weathering, erosion, and earthquakes.
	\item Environmental history: Every rock preserves evidence about the conditions under which it formed, offering a ``record'' of Earth's geologic past.
\end{itemize}

\section{What is Rock?}\label{sec:what-is-rock?}
\begin{itemize}
	\item \definition{Rock}{to a geologist, rock is coherent, naturally occurring solid that consists of an aggregate of minerals or, less common, a body of glass.}
	\begin{description}
		\item[Coherent] rocks hold together; they don't crumble easily.
		To separate them into smaller pieces, you have to physically break them.
		\item[Naturally occurring:] Only materials formed by natural geologic processes count as rocks (man-made concrete, for example, does not)
		\item[Aggregate of minerals or glass] Most rocks are collections of mineral grains or crystals, but some--like volcanic glass--consist of solid glass.
	\end{description}
\end{itemize}

\subsection{Coherent Solids}\label{subsec:coherent-solids}
\begin{itemize}
	\item Rocks are coherent solids \bc\ the grains \win\ them are bound together in one of two ways:
	\begin{description}
		\item[Clastic rocks (cememntation)] The grains are held together by natural cement.
		This cement is made of minerals that precipitated from water in the pore spaces between the grains.
		\item[Crystalline rocks (interlocking)] The mineral grains interlock \w\ each other, much like pieces of a jigsaw puzzle.
		This happens during the crystallization of minerals from a melt (igneous rocks) or recrystallization during metamorphism.
	\end{description}
\end{itemize}

\section{Chemical Composition of Rocks}\label{sec:chemical-composition-of-rocks}
Not all rocks contain the same chemicals.
\begin{description}
	\item[Silicate minerals] the most abundant:
	the majority of Earth's crust is made up of silicate minerals, which are built from silicon (\ce{Si}) and oxygen (\ce{O}).
	These two elements together make up almost 75\% of the crust's composition.
	\item[Carbonate minerals] Biogenic Contribution: at or near Earth's surface, living organisms play a significant role in rock formation.
	Many organisms play a significant role in rock formation.
	Many organisms extract calcium (\ce{Ca}) and carbonate (\ce{CO3^{2-}}) ions from water to form shells and skeletons.
	Over time, these materials accumulate and solidify into carbonate rocks such as limestone and dolostone.
\end{description}

\section{Rock Cycle}\label{sec:rock-cycle}
\begin{itemize}
	\item Although rocks may seem like permanent, unchanging masses, they are part of a continuous rock cycle that transforms one rock type into another over long periods of time.
	\item What drives the rock cycle?
	\begin{description}
		\item[Earth's internal heat] form igneous and metamorphic rocks.
		\item[Energy from the sun (external)] weathering and the transport of weathered materials.
	\end{description}
\end{itemize}

\section{Rock Classification}\label{sec:rock-classification}
In geology, rocks are classified based on how they form.
In the modern genetic scheme, there are three main rock classes:
\begin{description}
	\item[Igneous rocks] form by the solidification of molten rock, either magma or lava.
	\item[Sedimentary rocks] form either by cementing together grains broken off from pre-existing rocks or by precipitation of minerals from water solution at or near the Earth's surface.
	\item[Metamorphic rocks] form when pre-existing rocks change in texture or mineral composition due to increase in temperature, pressure, or stress.
\end{description}

\section{Physical Characteristics of Rocks}\label{sec:physical-characteristics-of-rocks}
Individual rock types are often distinguished from one another based on their physical characteristics, including:
\begin{itemize}
	\item Grain size and shape
	\item Rock composition
	\item Texture -- crystalline, clastic, or glassy.
	\item Layering -- e.g., bedding or foliation.
\end{itemize}

\section{Igneous Rocks}\label{sec:igneous-rocks}
\begin{itemize}
	\item Igneous rocks are formed by the freezing (solidification) of molten rock.
	They make up a great variety of rocks in the Earth's crust.
	\item How they form in nature:
	\begin{itemize}
		\item Hot molten rock fountaining from a volcanic opening may build up around the vent or flow downhill as lava.
		\item As lava flows, it can burn and melt the ground in its path.
		\item While cooling, the surface of the lava darkens and hardens, creating a crust that insulates the still-hot material beneath.
		\item Eventually, the lava stops moving and solidifies completely into a hard, black solid.
	\end{itemize}
\end{itemize}

\section{Formation of Melts}\label{sec:formation-of-melts}
Geologists use specific terms for molten rock:
\begin{description}
	\item[Magma] molten rock that remains beneath Earth's surface.
	\item[Lava] molten rock that reaches the Earth's surface.
	\item[Volcano] a vent through which lava emerges.
	When lava flows or explodes out of the vent, we call that a volcanic eruption.
	\item[Pyroclastic debris] the fragmental materials ejected during explosive eruption.
\end{description}
Geologist classify igneous rocks by where the melt solidifies:
\begin{itemize}
	\item Extrusive igneous rock
	\item Intrusive igneous rock
\end{itemize}

\section{Igneous Rock Classification}\label{sec:igneous-rock-classification}
\subsection{Extrusive Igneous Rock}\label{subsec:extrusive-igneous-rock}
\begin{itemize}
	\item Form when lava cools and solidifies above ground.
	\item Can also form by the cementing or welding togethering of pyroclastic debris.
\end{itemize}

\subsection{Inclusive Igneous Rock}\label{subsec:inclusive-igneous-rock}
\begin{itemize}
	\item Form when magma pushes into pre-existing rock below the surface and then cools and solidifies underground.
	\item Magma may accumulate in large, irregularly shaped magma chambers or intrude into rock as thin sheets or dikes.
\end{itemize}

\section{Chemical Variability of Molten Rock}\label{sec:chemical-variability-of-molten-rock}
Geologists distinguish among four major compositional types of molten rock depending on the proportion of silica relative to the combination of magnesium oxide and iron oxide that a melt contains.
\begin{description}
	\item[Mafic melts] contain a relatively high proportion of magnesium oxide and iron oxide relative to silica.
	\item[Ultramafic melts] have an even higher proportion of magnesium oxide and iron oxide relative to silica.
	\item[Felsic melts] have a relatively high proportion of silica relative to magnesium oxide and iron oxide.
	\item[Intermediate melts] are so named \bc\ their composition lies partway between those of mafic and felsic melts.
\end{description}

\begin{table}[H]
    \centering
    \begin{threeparttable}
		\caption{The Four Categories of Magma}
		\label{tab:four-magma-categories}
		\begin{tabular}{|l|l|}
			\hline
			Felsic (or silicic) magma & 67--76\% $\text{silica}^{*}$\\
			\hline
			Intermediate magma & 53--66\%\\
			\hline
			Mafic magma & 46--52\%\\
			\hline
			Ultramafic magma & 38--45\%\\
			\hline
		\end{tabular}
		\begin{tablenotes}
			\small
			\item $^{*}$ The numbers provided are `weight precent', meaning the proportion of the magma's weight that consist of silica.
		\end{tablenotes}
	\end{threeparttable}
\end{table}


\section{Factors Affecting the Compositions of Melts}\label{sec:factors-affecting-the-compositions-of-melts}
\begin{description}
	\item[Source-rock composition] the composition of a melt reflects the composition of the solid rock from which it was derived.
	\item[Partial melting] temperatures at sites of magma production typically do not become high enough to melt the entire source rock.
	Instead, magma tends to migrate away from the melting site before the rock has completely melted.
	As a result, only a portion of the original rock undergoes melting to produce magma.
	Magmas formed by partial melting are generally more felsic, while more silica-rich components often remain in the still-solid residue of the source rock.
	\item[Assimilation] as magma sits in a magma chamber before bully solidifying, it may incorporate chemical components from the surrounding wall rock.
	This can occur through the dissolution of wall rock into the magma or from blocks that break off, sink, and partially melt \win\ the magma body.
	\item[Magma mixing] magmas formed in different locations and from different source rocks may migrate into the same magma chamber, where they mix to form a hybrid melt \w\ a composition distinct from either original magma.
\end{description}

\section{Movement of Molten Rock}\label{sec:movement-of-molten-rock}
Why does magma rise?
Magma in the Earth rises for two reasons:
\begin{description}
	\item[Buoyancy] magma is less dense than the surrounding rock, so a buoyant force acts on it, driving it upward.
	\item[Pressure from overlying rock] the weight of the overlying rock creates pressure at depth, which squeezes the magma and forces it to move upward.
\end{description}
What controls the speed at which molten rock flows?
\begin{itemize}
	\item The speed of molten rock movement depends on its viscosity, or resistance to flow.
	Not all molten rock has the same viscosity--this property is primarily influenced by:
	\begin{description}
		\item[Temperature] High temperatures lower viscosity.
		\item[Volatile content] More volatiles (e.g., water, gases) lower viscosity.
		\item[Silica content] Higher silica increases viscosity, making the magma thicker and slower-moving.
	\end{description}
\end{itemize}

\section{Viscosity of Molten Rock}\label{sec:viscosity-of-molten-rock}
\begin{itemize}
	\item Mafic lava (less silica content)
	\begin{itemize}
		\item Mafic lava contains more volatiles and has relatively low viscosity.
		\item \Bc\ of its fluidity, it can erupt in fountains, travel long distances, and form thin, widespread lava flows.
	\end{itemize}
	\item Felsic to intermediate lava (higher silica content)
	\begin{itemize}
		\item Felsic lava is typically cooler, contains fewer volatiles, and has high viscosity.
		\item The high silica content results in more silicon-oxygen tetrahedra, which tend to form long chains that restrict flow.
		This structure makes felsic melt significantly more viscous than mafic melt.
		\item When felsic lava erupts, it often forms a lava dome -- a mound shaped accumulation of lava near the volcano's vent.
	\end{itemize}
\end{itemize}

\section{Solidification of Molten Rock}\label{sec:solidification-of-molten-rock}
Cooling occurs as magma rises toward the Earth's surface, where the surrounding crust is significantly cooler.
\begin{itemize}
	\item If magma becomes trapped underground as an intrusion, it gradually loses heat to the surrounding wall rock.
	When its temperature drops below the freezing point, it solidifies.
	\item If magma reaches the surface and erupts as lava, it cools rapidly due to contact \w\ much cooler air or water.
	\item In some cases, magma solidifies due to the loss of volatiles, which can reduce its ability to remain molten.
\end{itemize}

\section{Factors Affecting the Cooling Time of Magma}\label{sec:factors-affecting-the-cooling-time-of-magma}
\begin{itemize}
	\item Several factors influence how quickly magma cools and solidifies:
	\begin{description}
		\item[Depth of Intrusion] magma intruded deep \win\ the crust, where surrounding rock is also hot, cools more slowly than magma closer to the surface.
		\item[Shape and size of the magma body] heat escapes from the surface of a magma body.
		Therefore, for a given volume, the larger the surface area, the faster it cools.
		\item[The presence of circulating groundwater] groundwater flowing through the surrounding wall rock can carry away heat, much like a coolant in an engine, accelerating the cooling process.
	\end{description}
\end{itemize}

\section{Igneous Rocks Characterization}\label{sec:igneous-rocks-characterization}
\begin{itemize}
	\item Igneous rocks are described by the color and texture.
	\item Color generally reflects the rock's composition--influenced by grain size and by the presence of trace amounts of impurities.
	\item Textures reveal something about the history of how the melt cooled.
	\begin{itemize}
		\item Crystalline texture
		\item Fragmental texture
		\item Glassy texture
	\end{itemize}
\end{itemize}

\section{Igneous Rock's Texture}\label{sec:igneous-rock's-texture}
\begin{description}
	\item[Crystalline igneous rocks] consist of mineral crystals that intergrow when the melt solidifies, so that they fit together like jigsaw puzzle pieces.
	\begin{itemize}
		\item The interlocking of crystal in rocks happens \bc\ the rock does not solidify instantly.
		Rather different crystals grow at different rate and at different times, and they interface \w\ each other as the crystal grow.
	\end{itemize}
	\item[Fragmental igneous rocks] form pyroclastic debris and consist of igneous chunks, grains, or flakes that are packed together, welded together, or cemented together after they have solidified.
	\item[Glassy igneous rocks] are rocks made of a solid mass of glass or of glass surrounding isolated small crystals.
\end{description}

\section{Classifying Igneous Rocks}\label{sec:classifying-igneous-rocks}
\subsection{Crystalline Igneous Rocks}\label{subsec:crystalline-igneous-rocks}
\begin{itemize}
	\item The name applied ot a given rock sample depends on both the composition and the grain size of the sample.
	\item Color and density of an igneous rock provide clues to its composition:
	\begin{itemize}
		\item Felsic rock tend to be tan or pink/maroon and have low density.
		\item Mafic rock tend to be black or dark gray and have relatively high density.
	\end{itemize}
\end{itemize}

\subsection{Fragmented Igneous Rocks}\label{subsec:fragmented-igneous-rocks}
\begin{itemize}
	\item Accumulations of pyroclastic debris are called pyroclastic deposits.
	When these deposits consolidate into a solid mass, it becomes a pyroclastic rock.
	\item Geologists distinguish the types of pyroclastic rocks based on grain size:
	\begin{description}
		\item[Tuff (left)] fine-grained pyroclastic rock fragmented rock composed of ash
		\item[Volcanic breccia (right)] consists of angular fragments of pyroclastic debris
	\end{description}
\end{itemize}

\subsection{Glassy Igneous Rocks}\label{subsec:glassy-igneous-rocks}
\begin{itemize}
	\item Glassy texture most commonly in felsic igneous rocks \bc\ the high concentration of silica inhibits diffusion, and therefore, the growth of crystals.
	\item In some cases, a rapidly cooling lava freezes while it still contains gas bubbles; these bubbles remain as open holes known as \emph{vesicles}.
	\begin{description}
		\item[Obsidian] a vesicle-free felsic glass.
		\item[Tachylite] a relatively rare vesicle-free magic glass.
		\item[Pumice] a felsic volcanic rock that contains abundant vesicles (it floats on water).
		\item[Scoria] a mafic volcanic rock \w\ many vesicles.
	\end{description}
\end{itemize}

%</Lecture-9>
\end{document}
